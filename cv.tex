%%% LaTeX Template: Designer's CV
%%%
%%% Source: http://www.howtotex.com/
%%% Feel free to distribute this template, but please keep the referal to HowToTeX.com.
%%% Date: March 2012


%%%%%%%%%%%%%%%%%%%%%%%%%%%%%%%%%%%%%
% Document properties and packages
%%%%%%%%%%%%%%%%%%%%%%%%%%%%%%%%%%%%%
\documentclass[a4paper,12pt,final]{memoir}

% misc
\renewcommand{\familydefault}{bch}	% font
\pagestyle{empty}					% no pagenumbering
\setlength{\parindent}{0pt}			% no paragraph indentation


% required packages (add your own)
\usepackage{flowfram}										% column layout
\usepackage[top=1cm,left=1cm,right=1cm,bottom=1cm]{geometry}% margins
\usepackage{graphicx}										% figures
\usepackage{url}	
\usepackage{hyperref}										% URLs
\usepackage[usenames,dvipsnames]{xcolor}					% color
\usepackage{multicol}										% columns env.
	\setlength{\multicolsep}{0pt}
\usepackage{paralist}										% compact lists
\usepackage{tikz}

\definecolor{light-gray}{gray}{0.3}
\hyphenation{in-terested}

%%%%%%%%%%%%%%%%%%%%%%%%%%%%%%%%%%%%%
% Create column layout
%%%%%%%%%%%%%%%%%%%%%%%%%%%%%%%%%%%%%
% define length commands
\setlength{\vcolumnsep}{\baselineskip}
\setlength{\columnsep}{\vcolumnsep}

% frame setup (flowfram package)
% left frame
\newflowframe{0.3\textwidth}{\textheight}{0pt}{0pt}[left]
	\newlength{\LeftMainSep}
	\setlength{\LeftMainSep}{0.3\textwidth}
	\addtolength{\LeftMainSep}{1\columnsep}
 
% small static frame for the vertical line
\newstaticframe{1.5pt}{\textheight}{\LeftMainSep}{0pt}
 
% content of the static frame
\begin{staticcontents}{1}
\hfill
\tikz{%
	\draw[loosely dotted,color=MidnightBlue,line width=1.5pt,yshift=0]
	(0,0) -- (0,\textheight);}%
\hfill\mbox{}
\end{staticcontents}
 
% right frame
\addtolength{\LeftMainSep}{1.5pt}
\addtolength{\LeftMainSep}{1\columnsep}
\newflowframe{0.6\textwidth}{\textheight}{\LeftMainSep}{0pt}[main01]


%%%%%%%%%%%%%%%%%%%%%%%%%%%%%%%%%%%%%
% define macros (for convience)
%%%%%%%%%%%%%%%%%%%%%%%%%%%%%%%%%%%%%
\newcommand{\Sep}{\vspace{1.5em}}
\newcommand{\SmallSep}{\vspace{0.5em}}

\newenvironment{AboutMe}
	{\ignorespaces\textbf{\color{Black} About me}}
	{\Sep\ignorespacesafterend}
	
\newcommand{\CVSection}[1]
	{\Large\textbf{#1}\par
	\SmallSep\normalsize\normalfont}

\newcommand{\CVItem}[1]
	{\textbf{\color{MidnightBlue} #1}}


%%%%%%%%%%%%%%%%%%%%%%%%%%%%%%%%%%%%%
% Begin document
%%%%%%%%%%%%%%%%%%%%%%%%%%%%%%%%%%%%%
\begin{document}

% Left frame
%%%%%%%%%%%%%%%%%%%%
\begin{figure}
	\hfill
	\includegraphics[width=0.6\columnwidth]{photo.jpg}
	\vspace{-6cm}
\end{figure}

\begin{flushright}\small
	Bogoslavskyi Igor\\
	\textbf{Nationality:} \emph{\\Ukrainian\\}
	\textbf{Family:} \emph{\\single \\no children\\}
	\SmallSep
	\textbf{Phone:}\\
	+49 (0152) 04471543\\
	\textbf{Address:}\\ \emph{H\"{a}ndelstrasse 20, \\79104, Freiburg im Br. \\Germany}\\
	\SmallSep
	\textbf{In the Web:}
	\href{mailto:igor.bogoslavskyi@gmail.com}{igor.bogoslavskyi@gmail.com}\\ 
	\SmallSep
	\href{http://www.linkedin.com/pub/igor-bogoslavskyi/43/50b/726}{LinkedIn::Igor Bogoslavskyi}\\ 
	\SmallSep
	\href{https://github.com/niosus}{GitHub::niosus}\\
	\SmallSep
	\href{https://www.facebook.com/bogoslavskyi}{Facebook::bogoslavskyi}\\ 
	\SmallSep
	\href{https://plus.google.com/118159072920638602239/posts}{G+::Igor Bogoslavskyi}\\ 
	\SmallSep
\end{flushright}\normalsize
\framebreak


% Right frame
%%%%%%%%%%%%%%%%%%%%
\Huge\bfseries {\color{MidnightBlue} Igor Bogoslavskyi} \\
\Large\bfseries  Computer Science Student \\

\normalsize\normalfont

% About me
\begin{AboutMe}
\newline 
I am a young computer science specialist, particularly interested in Robotics, Machine Learning and Computer Vision fields of study. 

I am fond of the idea of autonomous moving vehicles and look forward to having a chance of facing the problems that arise in this field. \\
\emph{The best way to contact me would be just to send an e-mail}.
\end{AboutMe}

% Experience
\CVSection{Experience}
\CVItem{January 2012 - present, Assistant in AIS lab
\newline Albert Ludwigs University of Freiburg, Germany}
\begin{compactitem}[\color{RoyalBlue}$\circ$]
\item As an assistant in the Autonomous Intelligent Systems lab at Uni Freiburg, I dealt with Kinect RGBD sensors mounted onto various platforms as well as unmounted, with relation to RGBD object segmentation and traversability analysis. 
\end{compactitem}
\SmallSep

\CVItem{June 2012 - April 2013, Assistant in HRL lab
\newline Albert Ludwigs University of Freiburg, Germany}
\begin{compactitem}[\color{RoyalBlue}$\circ$]
	\item As an assistant in the Humanoid Robots Lab at Uni Freiburg, I dealt with Kinect RGBD sensors mounted onto the NAO robot platform with relation to gesture analysis. 
\end{compactitem}
\SmallSep

\CVItem{November 2011 - April 2012, Tutor in Image Processing class
\newline Albert Ludwigs University of Freiburg, Germany}
\begin{compactitem}[\color{RoyalBlue}$\circ$]
\item As a tutor in the chair of Computer Vision and Image Processing at Uni Freiburg my tasks were to help my fellow students to accomplish the course programming assignments. 
\item Involved c++ programming with relation to blurring/de-blurring, optical flow and segmentation tasks.
\end{compactitem}
\SmallSep

\CVItem{December 2010 - October 2011, Junior Software Developer
\newline Timecode LLC, Kyiv, Ukraine}
\begin{compactitem}[\color{MidnightBlue}$\circ$]
\item worked as part of a team on a game for Android platform. Involved Java Android programming, OpenGL.
\item worked as part of a team on an Online Content Store controlled via Kinect sensor. Mostly finding and fixing bugs. C\#.
\end{compactitem}
\Sep

% Education
\CVSection{Education}
\CVItem{2011 - present, Albert-Ludwigs-Universität Freiburg}\\
MSc. Applied Computer Science
\SmallSep

\CVItem{2007 - 2011, Kyiv National Taras Shevchenko University}\\
BSc. Faculty of Cybernetics. Applied Math field. 
\newline Chair of Computational Methods
\SmallSep

\CVItem{2004 - 2007, Lyceum 145, Kyiv}\\
Higher basic education certificate, Mathematics, Physics.
\Sep
\framebreak
\clearpage
\framebreak
\framebreak

% Skills
\CVSection{Skills}
\CVItem{Platforms and libraries}
\begin{multicols}{3}
\begin{compactitem}[\color{MidnightBlue}$\circ$]
	\item Linux
	\item Windows
	\item ROS 
	\item OpenCV
	\item QT
	\item OpenNI 
	\item OpenGL
	\item Android
	\item PCL
\end{compactitem}
\end{multicols}
\SmallSep

\CVItem{Programming and markup languages}
\begin{multicols}{3}
\begin{compactitem}[\color{MidnightBlue}$\circ$]
	\item C++ 
	\item Java 
	\item Python 
	\item Matlab/Octave 
	\item Xml
	\item SQLite
	\item CMake
\end{compactitem}
\end{multicols}
\SmallSep

\CVSection{Projects}
\CVItem{from October 2012, ROVINA Project}
\begin{compactitem}[\color{MidnightBlue}$\circ$]
	\item \href{http://www.rovina-project.eu/}{ROVINA}  - Robots for Exploration, Digital Preservation and Visualization of Archeological Sites.
	\item Traversability analysis based on depth images from Kinect-like sensor.
	\item Funded by European Commission.
	\item The corresponding paper was accepted to ECMR 2013 (European Conference on Mobile Robotics)
\end{compactitem}
\SmallSep

\CVSection{Trainings and Courses}
\CVItem{CITEC Robotics Summer School}
\begin{compactitem}[\color{MidnightBlue}$\circ$]
	\item Various presentations on robotics topic
	\item 5-day workshop on multi-sensoric data aquisition, focus on Kinect and eye-tracking cameras.
	\item \href{https://www.cit-ec.de/}{https://www.cit-ec.de/} 
\end{compactitem}
\SmallSep

\CVItem{AI Course (Online)}
\begin{compactitem}[\color{MidnightBlue}$\circ$]
	\item A course by Sebastian Thrun and Peter Norvig on AI.
\end{compactitem}
\SmallSep

\CVItem{AI for Robotics (Online)}
\begin{compactitem}[\color{MidnightBlue}$\circ$]
	\item A course by Sebastian Thrun on Udacity.
\end{compactitem}
\SmallSep

\CVItem{CUDA Programming (Online)}
\begin{compactitem}[\color{MidnightBlue}$\circ$]
	\item Yet not finished course on Udacity.
\end{compactitem}
\SmallSep

\CVSection{Publications}
\CVItem{Efficient Traversability Analysis for Mobile Robots using the Kinect Sensor}
\begin{compactitem}[\color{MidnightBlue}$\circ$]
	\item Presented at \href{http://www.iri.upc.edu/ecmr13/#home}{ECMR 2013} 
	\item The text of the paper can be found here: \\\href{http://www.informatik.uni-freiburg.de/~stachnis/pdf/bogoslavskyi13ecmr.pdf}{Efficient Traversability Analysis for Mobile Robots using the Kinect Sensor} 
\end{compactitem}
\SmallSep

\CVSection{Honors and Awards}
\CVItem{MINT Excellence Network Member}
\begin{compactitem}[\color{MidnightBlue}$\circ$]
	\item I was chosen as one of 300 best applicants to the \href{http://www.mlp.de/#/studenten/karriere/stipendienprogramme/mint-excellence}{MINT Excellence Network}. The candidates were chosen from the students who work in the fields of Math, Computer Science, Natural Sciences and Technic across Germany.
\end{compactitem}
\Sep

\framebreak
\clearpage
\framebreak
\framebreak

\CVSection{Languages}
\begin{compactitem}[\color{MidnightBlue}$\circ$]
	\item English (IELTS 8.0)
	\item German (B2+) 
	\item Russian (Native) 
	\item Ukrainian (Native)
\end{compactitem}
\Sep

\CVSection{Fields Of Interest}
\begin{compactitem}[\color{MidnightBlue}$\circ$]
	\item Autonomous Navigation
	\item Mobile Robotics
	\item Image Processing 
	\item Computer Vision 
	\item Machine Learning
	\item Mathematical Modelling
	\item Kinect 
	\item SLAM
	\item Android Programming
\end{compactitem}
\Sep 

\CVSection{Hobbies}
\begin{compactitem}[\color{MidnightBlue}$\circ$]
	\item Learning new things
	\item Android-programming
	\item Photography
	\item Volleyball
	\item Playing guitar 
	\item Rock music
	\item Skiing
	\item Biking
	\item New technology
\end{compactitem}
\Sep
% References
\CVSection{References}
References upon request.
\clearpage
\framebreak

%%%%%%%%%%%%%%%%%%%%%%%%%%%%%%%%%%%%%
% End document
%%%%%%%%%%%%%%%%%%%%%%%%%%%%%%%%%%%%%
\end{document}