%%% LaTeX Template: Designer's CV
%%%
%%% Source: http://www.howtotex.com/
%%% Feel free to distribute this template, but please keep the referal to HowToTeX.com.
%%% Date: March 2012


%%%%%%%%%%%%%%%%%%%%%%%%%%%%%%%%%%%%%
% Document properties and packages
%%%%%%%%%%%%%%%%%%%%%%%%%%%%%%%%%%%%%
\documentclass[a4paper,12pt,final]{memoir}

% misc
\renewcommand{\familydefault}{bch}	% font
\pagestyle{empty}					% no pagenumbering
\setlength{\parindent}{0pt}			% no paragraph indentation


% required packages (add your own)
\usepackage{flowfram}										% column layout
\usepackage[top=1cm,left=1cm,right=1cm,bottom=1cm]{geometry}% margins
\usepackage{graphicx}										% figures
\usepackage{url}
\usepackage{hyperref}										% URLs
\usepackage[usenames,dvipsnames]{xcolor}					% color
\usepackage{multicol}										% columns env.
	\setlength{\multicolsep}{0pt}
\usepackage{paralist}										% compact lists
\usepackage{tikz}

\definecolor{light-gray}{gray}{0.3}
\hyphenation{in-terested}

%%%%%%%%%%%%%%%%%%%%%%%%%%%%%%%%%%%%%
% Create column layout
%%%%%%%%%%%%%%%%%%%%%%%%%%%%%%%%%%%%%
% define length commands
\setlength{\vcolumnsep}{\baselineskip}
\setlength{\columnsep}{\vcolumnsep}

% frame setup (flowfram package)
% left frame
\newflowframe{0.35\textwidth}{\textheight}{0pt}{0pt}[left]
	\newlength{\LeftMainSep}
	\setlength{\LeftMainSep}{0.35\textwidth}
	\addtolength{\LeftMainSep}{1\columnsep}

% small static frame for the vertical line
\newstaticframe{1.5pt}{\textheight}{\LeftMainSep}{0pt}

% content of the static frame
\begin{staticcontents}{1}
\hfill
\tikz{%
	\draw[loosely dotted,color=MidnightBlue,line width=1.5pt,yshift=0]
	(0,0) -- (0,\textheight);}%
\hfill\mbox{}
\end{staticcontents}

% right frame
\addtolength{\LeftMainSep}{1.5pt}
\addtolength{\LeftMainSep}{1\columnsep}
\newflowframe{0.6\textwidth}{\textheight}{\LeftMainSep}{0pt}[main01]


%%%%%%%%%%%%%%%%%%%%%%%%%%%%%%%%%%%%%
% define macros (for convience)
%%%%%%%%%%%%%%%%%%%%%%%%%%%%%%%%%%%%%
\newcommand{\Sep}{\vspace{1.5em}}
\newcommand{\SmallSep}{\vspace{0.5em}}

\newenvironment{AboutMe}
	{\ignorespaces\textbf{\color{Black} About me}}
	{\Sep\ignorespacesafterend}

\newcommand{\CVSection}[1]
	{\Large\textbf{#1}\par
	\SmallSep\normalsize\normalfont}

\newcommand{\CVItem}[1]
	{\textbf{\color{MidnightBlue} #1}}


%%%%%%%%%%%%%%%%%%%%%%%%%%%%%%%%%%%%%
% Begin document
%%%%%%%%%%%%%%%%%%%%%%%%%%%%%%%%%%%%%
\begin{document}

% Left frame
%%%%%%%%%%%%%%%%%%%%
\begin{figure}
	\hfill
	\includegraphics[width=\columnwidth]{igor.jpg}
	\vspace{-5cm}
\end{figure}

\begin{flushleft}\small
	\textbf{Name:} Bogoslavskyi Igor\\
	\SmallSep
	\textbf{Nationality:} \emph{Ukrainian}\\
	\SmallSep
	\textbf{Family:} \emph{single, no children}\\
	\SmallSep
	\textbf{Phone:} \\ +49 (152) 04471543\\
	\SmallSep
	\textbf{Address:}\\ \emph{Ellerstr. 33, \\53119, Bonn \\Germany}\\
	\SmallSep
	\textbf{On the Web:}
	\href{mailto:igor.bogoslavskyi@gmail.com}{igor.bogoslavskyi@gmail.com}\\
	\SmallSep
	\href{https://de.linkedin.com/in/igor-bogoslavskyi-72650b43}{LinkedIn::Igor Bogoslavskyi}\\
	\SmallSep
	\href{https://github.com/niosus}{GitHub::niosus}\\
	\SmallSep
\end{flushleft}
\normalsize

\framebreak{}


% Right frame
%%%%%%%%%%%%%%%%%%%%
\Huge\bfseries {\color{MidnightBlue} Igor Bogoslavskyi} \\
\Large\bfseries  PhD Candidate \\

\normalsize\normalfont{}

% About me
\begin{AboutMe}
\newline
Igor Bogoslavskyi is a PhD student at the lab for photogrammetry at the
University of Bonn led by Cyrill Stachniss. Before moving to Bonn, he has
finished his Master of Science studies at the University of Freiburg in Germany
in 2011 and Bachelor of Science in Ukraine in 2007. During his master studies he
was working as a lab assistant on ROVINA project in AIS laboratory led by
Wolfram Burgard. His current interests lie in scene interpretation and outdoor
perception and navigation for mobile robots.
\end{AboutMe}

% Experience
\CVSection{Experience}
\CVItem{2014 --- Present, PhD candidate, Photogrammetry lab
\newline Rheinische Friedrich-Wilhelms University Bonn, Germany}
\begin{compactitem}[\color{RoyalBlue}$\circ$]
\item I am now a PhD candidate in the institute of Geodesy, Geoinformation and
Cartography at the University of Bonn. My advisor is Prof.~Cyrill Stachniss.
\end{compactitem}
\SmallSep

\CVItem{2012 --- 2014, Assistant, AIS lab
\newline Albert Ludwigs University of Freiburg, Germany}
\begin{compactitem}[\color{RoyalBlue}$\circ$]
\item As an assistant in the Autonomous Intelligent Systems lab at Uni Freiburg,
I dealt with Kinect RGBD sensors mounted onto various platforms. I have
implemented traversability analysis for a mobile robot as part of ROVINA
project. The developments in this project let to publications at ECMR'13 and
ICRA'16 conferences.
\end{compactitem}
\SmallSep

\CVItem{2012 --- 2013, Assistant, HRL lab
\newline Albert Ludwigs University of Freiburg, Germany}
\begin{compactitem}[\color{RoyalBlue}$\circ$]
	\item As an assistant in the Humanoid Robots Lab at Uni Freiburg, I dealt with
	Kinect RGBD sensors mounted onto the NAO robot platform and have implemented a
	system that detected human pointing gestures generating a goal for a robot.
\end{compactitem}
\SmallSep

\CVItem{2011 --- 2012, Tutor, Image Processing course
\newline Albert Ludwigs University of Freiburg, Germany}
\begin{compactitem}[\color{RoyalBlue}$\circ$]
\item During my first semester at Freiburg University I have been tutoring in
the chair of Computer Vision and Image Processing. My task was to help my fellow
students to accomplish the course programming assignments.
\end{compactitem}
\SmallSep

\CVItem{2010 --- 2011, Junior Software Developer
\newline Timecode LLC, Kyiv, Ukraine}
\begin{compactitem}[\color{MidnightBlue}$\circ$]
\item worked as part of a team on a game for Android platform. Java Android
programming, OpenGL.\@
\item worked as part of a team on an Online Content Store controlled via Kinect
sensor. C\#.
\end{compactitem}
\SmallSep
\framebreak
\clearpage
\framebreak{}
	% LEFT FRAME 2nd PAGE
	\SmallSep{}
	\vspace{-2mm}

	\CVSection{I Mostly Code In:}
	\begin{compactitem}[\color{MidnightBlue}$\circ$]
		\item C\texttt{++}
		\item Python
		\item Java
		\item Matlab/Octave
	\end{compactitem}
	\Sep{}

	\CVSection{Languages}
	\begin{compactitem}[\color{MidnightBlue}$\circ$]
		\item English (IELTS 8.0)
		\item German (B2+)
		\item Russian (Native)
		\item Ukrainian (Native)
	\end{compactitem}
	\Sep{}

	\CVSection{Honors and Awards}
	\CVItem{MINT Excellence Network Member}
	\begin{compactitem}[\color{MidnightBlue}$\circ$]
		\item I was chosen as one of 300 best applicants across Germany to the
		\href{http://www.mlp.de/#/studenten/karriere/stipendienprogramme/mint-excellence}
		{MINT Excellence Network}.

		The candidates were chosen from the students who work in the fields of Math,
		Computer Science, Natural Sciences and Tech across Germany.
	\end{compactitem}
	\SmallSep{}

	\CVSection{Fields Of Interest}
	\begin{compactitem}[\color{MidnightBlue}$\circ$]
		\item Probabilistic Robotics
		\item Autonomous Outdoor Navigation
		\item Scene Interpretation
		\item Dynamics Detection
		\item Machine Learning
		\item SLAM
	\end{compactitem}
	\SmallSep

	\vfill
	\vspace{7.3cm}
	% References
	\CVSection{References}
		References upon request.
	\Sep

\vfill
\framebreak{}

% Education
\CVSection{Education}
\CVItem{2014 --- Current, Friedrich-Wilhelms-Universit\"at Bonn}\\
PhD in photogrammetry and mobile robotics
\SmallSep{}

\CVItem{2011 --- 2014, Albert-Ludwigs-Universit\"at Freiburg}\\
MSc. Applied Computer Science. Final grade: excellent.
\SmallSep{}

\CVItem{2007 --- 2011, Kyiv National Taras Shevchenko University}\\
BSc. Faculty of Cybernetics. Applied Math field.
\newline Chair of Computational Methods
\SmallSep{}

\CVItem{2004 --- 2007, Lyceum 145, Kyiv, Ukraine}\\
Higher basic education certificate, Mathematics, Physics.
\SmallSep{}


\CVSection{Notable Projects}
\CVItem{2012 --- 2016, \href{http://www.rovina-project.eu/}{ROVINA} Project}
\begin{compactitem}[\color{MidnightBlue}$\circ$]
	\item Presents an autonomous robot for underground exploration.
	\item Components implemented by me in C\texttt{++}:
	\begin{compactitem}
	 	\item traversability analysis for the robot.
	 	\item a robust homing algorithm to return robot home
		\item most of exploration and navigation stack of the robot
	 \end{compactitem}
	\item Project has received excellent reviews from EU commission.
	\item My papers were accepted to ECMR'13 and ICRA'16.
\end{compactitem}
\CVItem{2016 --- Current, \href{https://github.com/niosus/EasyClangComplete}{EasyClangComplete}} Project
\begin{compactitem}[\color{MidnightBlue}$\circ$]
	\item It is an easy-to-setup plugin for C++ autocompletion.
	\item Currently 140 stars on GitHub.
\end{compactitem}
\SmallSep

\CVSection{Publications}
\CVItem{Fast range image-based segmentation of sparse 3d laser scans for online operation}
\begin{compactitem}[\color{MidnightBlue}$\circ$]
	\item Presented at \href{http://iros2016.org/}{IROS 2016}
	\item An approach to segment single Velodyne-produced point clouds much faster then sensor frame-rate.
\end{compactitem}
\CVItem{Robust homing for autonomous robots}
\begin{compactitem}[\color{MidnightBlue}$\circ$]
	\item Presented at \href{http://icra2016.org/}{ICRA 2016}
	\item A robust homing approach for an autonomous robot exploring underground environments.
\end{compactitem}
\CVItem{Where to Park? Minimizing the Expected Time to Find a Parking Space}
\begin{compactitem}[\color{MidnightBlue}$\circ$]
	\item Presented at \href{http://icra2015.org/}{ICRA 2015}
	\item An MDP based approach to minimize the expected time to find a parking spot and reach the goal by foot.
\end{compactitem}
\CVItem{Efficient Traversability Analysis for Mobile Robots using the Kinect Sensor}
\begin{compactitem}[\color{MidnightBlue}$\circ$]
	\item Presented at \href{http://www.iri.upc.edu/ecmr13/#home}{ECMR 2013}
	\item A fast and reliable traversability analysis algorithm for a robot
	operating in underground environments.
\end{compactitem}
\SmallSep

\vfill
\CVSection{\hspace{60mm}Igor Bogoslavskyi}
\vspace{3mm}
Date: \hspace{28mm} Signature:
\clearpage
\framebreak

%%%%%%%%%%%%%%%%%%%%%%%%%%%%%%%%%%%%%
% End document
%%%%%%%%%%%%%%%%%%%%%%%%%%%%%%%%%%%%%
\end{document}
