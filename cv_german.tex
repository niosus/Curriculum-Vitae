%%% LaTeX Template: Designer's CV
%%%
%%% Source: http://www.howtotex.com/
%%% Feel free to distribute this template, but please keep the referal to HowToTeX.com.
%%% Date: March 2012


%%%%%%%%%%%%%%%%%%%%%%%%%%%%%%%%%%%%%
% Document properties and packages
%%%%%%%%%%%%%%%%%%%%%%%%%%%%%%%%%%%%%
\documentclass[a4paper,12pt,final]{memoir}


% misc
\renewcommand{\familydefault}{bch}	% font
\pagestyle{empty}					% no pagenumbering
\setlength{\parindent}{0pt}			% no paragraph indentation


% required packages (add your own)
\usepackage{flowfram}										% column layout
\usepackage[top=1cm,left=1cm,right=1cm,bottom=1cm]{geometry}% margins
\usepackage{graphicx}										% figures
\usepackage{url}	
\usepackage{hyperref}										% URLs
\usepackage[usenames,dvipsnames]{xcolor}					% color
\usepackage{multicol}										% columns env.
	\setlength{\multicolsep}{0pt}
\usepackage{paralist}										% compact lists
\usepackage{tikz}

\definecolor{light-gray}{gray}{0.3}
\hyphenation{in-terested}

%%%%%%%%%%%%%%%%%%%%%%%%%%%%%%%%%%%%%
% Create column layout
%%%%%%%%%%%%%%%%%%%%%%%%%%%%%%%%%%%%%
% define length commands
\setlength{\vcolumnsep}{\baselineskip}
\setlength{\columnsep}{\vcolumnsep}

% frame setup (flowfram package)
% left frame
\newflowframe{0.25\textwidth}{\textheight}{0pt}{0pt}[left]
	\newlength{\LeftMainSep}
	\setlength{\LeftMainSep}{0.25\textwidth}
	\addtolength{\LeftMainSep}{1\columnsep}
 
% small static frame for the vertical line
\newstaticframe{1.5pt}{\textheight}{\LeftMainSep}{0pt}
 
% content of the static frame
\begin{staticcontents}{1}
\hfill
\tikz{%
	\draw[loosely dotted,color=MidnightBlue,line width=1.5pt,yshift=0]
	(0,0) -- (0,\textheight);}%
\hfill\mbox{}
\end{staticcontents}
 
% right frame
\addtolength{\LeftMainSep}{1.5pt}
\addtolength{\LeftMainSep}{1\columnsep}
\newflowframe{0.65\textwidth}{\textheight}{\LeftMainSep}{0pt}[main01]


%%%%%%%%%%%%%%%%%%%%%%%%%%%%%%%%%%%%%
% define macros (for convience)
%%%%%%%%%%%%%%%%%%%%%%%%%%%%%%%%%%%%%
\newcommand{\Sep}{\vspace{1.5em}}
\newcommand{\SmallSep}{\vspace{0.5em}}

\newenvironment{AboutMe}
	{\ignorespaces\textbf{\color{Black} \"{U}ber mich}}
	{\Sep\ignorespacesafterend}
	
\newcommand{\CVSection}[1]
	{\Large\textbf{#1}\par
	\SmallSep\normalsize\normalfont}

\newcommand{\CVItem}[1]
	{\textbf{\color{MidnightBlue} #1}}


%%%%%%%%%%%%%%%%%%%%%%%%%%%%%%%%%%%%%
% Begin document
%%%%%%%%%%%%%%%%%%%%%%%%%%%%%%%%%%%%%
\begin{document}

% Left frame
%%%%%%%%%%%%%%%%%%%%
\begin{figure}
	\hfill
	\includegraphics[width=0.6\columnwidth]{photo.jpg}
	\vspace{-6cm}
\end{figure}
\begin{flushright}\small
	Bogoslavskyi Igor
	\textbf{Nationalit\"{a}t:} \emph{\\ukrainisch\\}
	\textbf{Familienstand:} \emph{\\ledig \\keine Kinder\\}
	\SmallSep
	\textbf{Telefon:}\\
	+49 (0152) 04471543\\
	\textbf{Adresse:}\\ \emph{H\"{a}ndelstrasse 20, \\79104, Freiburg im Br. \\Deutschland}\\
	\SmallSep
	\textbf{Ich im Netz:}
	\href{mailto:igor.bogoslavskyi@gmail.com}{igor.bogoslavskyi@gmail.com}\\ 
	\SmallSep
	\href{http://www.linkedin.com/pub/igor-bogoslavskyi/43/50b/726}{LinkedIn::Igor Bogoslavskyi}\\ 
	\SmallSep
	\href{https://github.com/niosus}{GitHub::niosus}\\
	\SmallSep
	\href{https://www.facebook.com/bogoslavskyi}{Facebook::bogoslavskyi}\\ 
	\SmallSep
\end{flushright}\normalsize
\framebreak


% Right frame
%%%%%%%%%%%%%%%%%%%%
\Huge\bfseries {\color{MidnightBlue} Igor Bogoslavskyi} \\
\Large\bfseries  Informatik Student \\

\normalsize\normalfont

% About me
\begin{AboutMe}
\newline 
Ich bin einer junge Informatik-Spezialist, insbesondere in der Robotik und Computer Vision Fachrichtungen interessiert.

Ich bin begeistert von der Idee des selbstfahrenden Autos und freue mich auf eine Chance den Problemen, die in diesem Bereich entstehen, zu l\"{o}sen. \\
\emph {Der beste Weg, mich zu kontaktieren w\"{a}re einfach eine E-Mail senden.}
\end{AboutMe}

% Experience
\CVSection{Erfahrung}
\CVItem{Juni 2012 - April 2013, Hi-Wi im HRL Labor
\newline Albert Ludwigs Universit\"{a}t Freiburg, Deutschland}
\begin{compactitem}[\color{RoyalBlue}$\circ$]
	\item Als Hi-Wi im Humanoid Robots Labor an der Uni Freiburg, habe ich mit RGBD Sensoren auf dem NAO Roboter gearbeitet. Meine Aufgabe war eine Methode einf\"{u}hren um weisenden Gesten zu erkennen. 
\end{compactitem}
\SmallSep

\CVItem{seit Februar 2012, Hi-Wi im AIS Labor
\newline Albert Ludwigs Universit\"{a}t Freiburg, Deutschland}
\begin{compactitem}[\color{RoyalBlue}$\circ$]
\item Als Hi-Wi im Autonomous Intelligent Systems Labor an der Uni Freiburg, habe ich mit Kinect RGBD Sensoren gearbeitet. Meine Aufgaben waren Bildaufteilung und Traversierbarkeitsanalyse basierend auf den Tiefbildern. 
\end{compactitem}
\SmallSep

\CVItem{November 2011 - April 2012, Tutor im Bildverarbeitung Kurs
\newline Albert Ludwigs Universit\"{a}t Freiburg, Deutschland}
\begin{compactitem}[\color{RoyalBlue}$\circ$]
\item Als Tutor am Lehrstuhl f\"{u}r Computer Vision und Image Processing an der Uni Freiburg meine Aufgaben waren den Studenten mit ihren Programmieraufgaben zu helfen. 
\item Beteiligte c++ Programmierung mit Bezug zu blurring/de-blurring, optischer Fluss und Segmentierung Aufgaben.
\end{compactitem}
\SmallSep

\CVItem{December 2010 - Oktober 2011, Junior Software Developer
\newline Timecode LLC, Kiew, Ukraine}
\begin{compactitem}[\color{MidnightBlue}$\circ$]
\item arbeitete als Teil eines Teams auf ein Spiel f\"{u}r Android-Plattform. Beteiligte Java Android Programmierung. OpenGL.
\item arbeitete als Teil eines Teams auf einer via Kinect-Sensor gesteuerter Online Content Store. Meistens bug-fixing. C\#.
\end{compactitem}
\SmallSep

% Education
\CVSection{Ausbildung}
\CVItem{seit Oktober 2011, Albert Ludwigs Universit\"{a}t Freiburg}\\
MSc. Angewandte Informatik
\SmallSep

\CVItem{2007 - 2011, Kyiv National Taras Shevchenko University}\\
BSc. Fakult\"{a}t f\"{u}r Kybernetik. Angewandte Mathematik. 
\newline Stuhl von Rechenmethoden.
\SmallSep

\CVItem{2004 - 2007, Lyzeum 145, Kiew}\\
H\"{o}here Grundbildung Zertifikat, Mathematik, Physik.
\SmallSep
\framebreak
\clearpage
\framebreak
\framebreak

\CVSection{Projekte}
\CVItem{seit Oktober 2012, ROVINA Project}
\begin{compactitem}[\color{MidnightBlue}$\circ$]
	\item ROVINA - Robots for Exploration, Digital Preservation and Visualization of Archeological Sites
	\item \href{http://www.rovina-project.eu/}{http://www.rovina-project.eu/} 
	\item Traversierbarkeitsanalyse basierend auf Tiefbildern
	\item Gef\"{o}rdert von der Europ\"{a}ischen Kommission
\end{compactitem}
\SmallSep

\CVSection{Sprachen}
\begin{compactitem}[\color{MidnightBlue}$\circ$]
	\item Englisch (IELTS 8.0)
	\item Deutsch (~B2) 
	\item Russisch (Muttersprache) 
	\item Ukrainisch (Muttersprache)
\end{compactitem}
\SmallSep

% Skills
\CVSection{F\"{a}higkeiten}
\CVItem{Plattformen und Bibliotheken}
\begin{multicols}{3}
\begin{compactitem}[\color{MidnightBlue}$\circ$]
	\item ROS 
	\item OpenCV
	\item OpenNI 
	\item OpenGL
	\item Android
	\item PCL
\end{compactitem}
\end{multicols}
\SmallSep

\CVItem{Programmier- und Markup-Sprachen}
\begin{multicols}{3}
\begin{compactitem}[\color{MidnightBlue}$\circ$]
	\item C++ 
	\item Java 
	\item Python 
	\item Matlab/Octave 
	\item Xml
	\item SQLite
\end{compactitem}
\end{multicols}
\SmallSep

\CVItem{Interessenschwerpunkte}
\begin{compactitem}[\color{MidnightBlue}$\circ$]
	\item Mobile Robotics
	\item Image Processing 
	\item Kinect 
	\item Computer Vision 
	\item Autonomous Navigation
	\item SLAM
\end{compactitem}
\SmallSep

\CVSection{Hobbys}
\begin{compactitem}[\color{MidnightBlue}$\circ$]
	\item Android-programmierung
	\item Volleyball
	\item Gitarre 
	\item Rock Musik
	\item Skifahren
	\item Hiking
	\item Biking
	\item Technologie
	\item Neue Dinge lernen
\end{compactitem}
\SmallSep

% References
\CVSection{Referenzen}
Referenzen auf Anfrage.

\clearpage
\framebreak

%%%%%%%%%%%%%%%%%%%%%%%%%%%%%%%%%%%%%
% End document
%%%%%%%%%%%%%%%%%%%%%%%%%%%%%%%%%%%%%
\end{document}